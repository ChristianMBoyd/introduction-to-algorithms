\subsection{Designing algorithms}

\subsubsection{}
    The initial sequence $[3, 41, 52, 26, 38, 57, 9, 49]$ will be merge-sorted according to the following steps.
    \begin{eqnarray}
        \nonumber &\,& [3, 41, 52, 26, 38, 57, 9, 49] \to 
            [3, 41, 52, 26] [38, 57, 9, 49]
        \\ \nonumber &\,& \to
        [3, 41] [52, 26] [38, 57] [9, 49] \to
        [3, 41] [26, 52] [38, 57] [9, 49]
        \\ \nonumber &\,& \to
        [3, 26, 41, 52] [9, 38, 49, 57] \to
        [3, 9, 26, 38, 41, 49, 52, 57]
    \end{eqnarray}

\subsubsection{}
    Suppose that MergeSort is invoked only as MergeSort$(A, 1, n)$ in client code for $n\ge1$.  Let's now change line 1 to test for {\bf if }$p \neq r$.  In the case where $n=1$, we hit this test and correctly return the sorted, single-item list.  If $n>1$, then $q:=\left\lfloor (1 + n)/2\right\rfloor$ and the recursive call on line 4, MergeSort$(A, p, q)$ with $p=1$, is of the form already considered in the problem statement.  The only concern is the second recursive call on line 5, MergeSort $(A, q + 1, r)$ (with $r=n$ currently).  

    The question now becomes: if $p\le r$, can it be that $q + 1 > r$?  If $p=r$, then we will abort at the test in line 1.  If $p<r$, 
    \begin{equation}
        q:=\left\lfloor \frac{p + r}{2} \right\rfloor \le  \left\lfloor \frac{2 r - 1}{2} \right\rfloor = r - 1\,\,\,.
    \end{equation}
    As a result $q+1\le r$ and the recursive call on line 5 will never result in a call to MergeSort where $p>r$.  We can safely switch the test on line 1 to $p\neq r$ without concern so long as client code abides by the problem statement.